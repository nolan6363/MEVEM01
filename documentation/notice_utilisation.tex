\documentclass[12pt,a4paper]{article}
\usepackage[utf8]{inputenc}
\usepackage[french]{babel}
\usepackage[T1]{fontenc}
\usepackage{geometry}
\usepackage{graphicx}
\usepackage{fancyhdr}
\usepackage{amsmath}
\usepackage{amsfonts}
\usepackage{amssymb}
\usepackage{xcolor}
\usepackage{hyperref}
\usepackage{listings}
\usepackage{float}
\usepackage{caption}
\usepackage{subcaption}
\usepackage{booktabs}
\usepackage{array}
\usepackage{multirow}
\usepackage{longtable}

% Configuration des pages
\geometry{left=2.5cm,right=2.5cm,top=2.5cm,bottom=2.5cm}
\pagestyle{fancy}
\fancyhf{}
\rhead{\thepage}
\lhead{MEVEM - Notice d'utilisation}

% Configuration des liens
\hypersetup{
    colorlinks=true,
    linkcolor=blue,
    filecolor=magenta,      
    urlcolor=cyan,
    pdftitle={MEVEM - Notice d'utilisation},
    pdfauthor={MEVEM Team}
}

% Configuration des listings de code
\lstset{
    basicstyle=\ttfamily\small,
    breaklines=true,
    frame=single,
    numbers=left,
    numberstyle=\tiny,
    numbersep=5pt,
    backgroundcolor=\color{gray!10},
    keywordstyle=\color{blue},
    commentstyle=\color{green!60!black},
    stringstyle=\color{red}
}

\title{%
    \vspace{-2cm}
    {\Huge \textbf{MEVEM}}\\[0.5cm]
    {\Large Mesure de la verse du maïs}\\[0.3cm]
    {\Large \textbf{NOTICE D'UTILISATION}}\\[0.5cm]
    {\normalsize Version 1.0}
}
\author{Nolan Bayon}
\date{\today}

\begin{document}

\maketitle
\thispagestyle{empty}

\newpage
\tableofcontents
\newpage

% =====================================
% PRÉSENTATION GÉNÉRALE
% =====================================
\section{Présentation générale}

\subsection{Vue d'ensemble}

MEVEM (Mesure de la verse du maïs) est un système complet de mesure et d'analyse de la résistance à la verse des plants de maïs. Le système combine :

\begin{itemize}
    \item Une interface mécanique de mesure avec capteurs d'angle et de force
    \item Un logiciel d'acquisition et d'analyse en temps réel
    \item Un système de calibration intégré
    \item Des outils d'export et d'analyse des données
\end{itemize}

\subsection{Principe de fonctionnement}

Le système MEVEM mesure simultanément :
\begin{itemize}
    \item \textbf{L'angle d'inclinaison} du plant lors de l'application d'une force (capteur d'angle)
    \item \textbf{La force appliquée} nécessaire pour incliner le plant (capteur de force)
\end{itemize}

Ces mesures permettent d'établir des courbes force-angle caractéristiques de la résistance à la verse.

\subsection{Contenu de la livraison}

\begin{itemize}
    \item Dispositif de mesure MEVEM avec capteurs intégrés
    \item Câble USB de connexion
    \item Logiciel MEVEM (exécutable Windows/Linux)
    \item Documentation technique et notice
\end{itemize}

% =====================================
% INSTALLATION ET CONFIGURATION
% =====================================
\section{Installation et configuration}

\subsection{Prérequis système}

\subsubsection{Configuration minimale}
\begin{itemize}
    \item \textbf{Système d'exploitation :} Windows 10+ ou Linux Ubuntu 18.04+
    \item \textbf{Mémoire :} 4 GB RAM minimum
    \item \textbf{Espace disque :} 200 MB disponibles
    \item \textbf{Connectivité :} Port USB 2.0 disponible
    \item \textbf{Navigateur :} Chrome, Firefox, Safari ou Edge (version récente)
\end{itemize}

\subsection{Installation du logiciel}

\subsubsection{Installation rapide (recommandée)}

\begin{enumerate}
    \item Téléchargez l'exécutable MEVEM correspondant à votre système :
    \begin{itemize}
        \item \texttt{mevem\_windows.exe} pour Windows
        \item \texttt{mevem\_linux} pour Linux
    \end{itemize}
    
    \item Placez l'exécutable dans un dossier dédié
    
    \item \textbf{Linux uniquement :} Rendez le fichier exécutable :
    \begin{lstlisting}[language=bash]
chmod +x mevem_linux
    \end{lstlisting}
\end{enumerate}

\subsubsection{Configuration des permissions (Linux)}

Sur Linux, il est nécessaire de configurer les permissions d'accès aux ports série :

\begin{enumerate}
    \item Ajoutez votre utilisateur au groupe \texttt{dialout} :
    \begin{lstlisting}[language=bash]
sudo usermod -a -G dialout $USER
    \end{lstlisting}
    
    \item Redémarrez votre session (déconnexion/reconnexion)
    
    \item Vérifiez la configuration :
    \begin{lstlisting}[language=bash]
groups $USER
    \end{lstlisting}
\end{enumerate}

\subsection{Connexion du matériel}

\begin{enumerate}
    \item Connectez le dispositif MEVEM à votre ordinateur via le câble USB fourni
    \item Le système détecte automatiquement les ports série disponibles
    \item Les ports testés automatiquement sont :
    \begin{itemize}
        \item \textbf{Windows :} COM3, COM4, COM5
        \item \textbf{Linux :} /dev/ttyUSB0, /dev/ttyUSB1, /dev/ttyACM0
    \end{itemize}
\end{enumerate}

% =====================================
% PREMIÈRE UTILISATION
% =====================================
\section{Première utilisation}

\subsection{Démarrage du système}

\begin{enumerate}
    \item Lancez l'exécutable MEVEM :
    \begin{itemize}
        \item \textbf{Windows :} Double-cliquez sur \texttt{mevem\_windows.exe}
        \item \textbf{Linux :} Exécutez \texttt{./mevem\_linux} dans un terminal
    \end{itemize}
    
    \item L'interface web s'ouvre automatiquement dans votre navigateur par défaut
    \item URL d'accès manuel : \url{http://127.0.0.1:5000}
\end{enumerate}

\subsection{Interface utilisateur}

\subsubsection{Panneau de contrôle}

Le panneau de contrôle situé en haut de l'interface comprend :

\begin{itemize}
    \item \textbf{Démarrer mesure :} Lance l'acquisition de données
    \item \textbf{Arrêter mesure :} Arrête l'acquisition en cours
    \item \textbf{Effacer données :} Supprime les données actuelles
    \item \textbf{Exporter Excel :} Sauvegarde les données au format Excel
    \item \textbf{Calibrer capteurs :} Lance la procédure de calibration
    \item \textbf{Actualiser statut :} Rafraîchit l'état de connexion
\end{itemize}

\subsubsection{Zone d'affichage}

\begin{itemize}
    \item \textbf{Graphique principal :} Courbe force vs angle en temps réel
    \item \textbf{Panneau d'information :} État de connexion et calibration
    \item \textbf{Statistiques :} Durée, amplitude, force maximale
\end{itemize}

% =====================================
% CALIBRATION
% =====================================
\section{Calibration des capteurs}

\subsection{Importance de la calibration}

La calibration est \textbf{obligatoire} avant toute mesure. Elle permet :
\begin{itemize}
    \item D'assurer la précision des mesures
    \item De convertir les valeurs brutes des capteurs en unités physiques
    \item De compenser les variations liées à l'environnement
\end{itemize}

\subsection{Procédure de calibration}

\subsubsection{Préparation}

Avant de commencer la calibration, assurez-vous de disposer de :
\begin{itemize}
    \item Étalons d'angle : supports à 0° et 45°
    \item Masse étalon : poids de 1 kg ± 10 g
    \item Surface plane et stable
\end{itemize}

\subsubsection{Étapes de calibration}

\begin{enumerate}
    \item Cliquez sur le bouton \textbf{Calibrer capteurs}
    
    \item \textbf{Calibration de l'angle :}
    \begin{enumerate}
        \item Placez le dispositif en position horizontale (0°)
        
        \item Appuyez sur "Entrée" quand la position est stable
        \item Inclinez le dispositif à exactement 45°
        
        \item Appuyez sur "Entrée" quand la position est stable
    \end{enumerate}
    
    \item \textbf{Calibration de la force :}
    \begin{enumerate}
        \item Retirez toute charge du capteur (position à vide)
        
        \item Appuyez sur "Entrée" pour enregistrer le zéro
        \item Placez exactement 1 kg sur le capteur
        
        \item Appuyez sur "Entrée" pour finaliser
    \end{enumerate}
    
    \item La calibration est automatiquement sauvegardée
\end{enumerate}

\subsection{Vérification de la calibration}

Après calibration, vérifiez que :
\begin{itemize}
    \item Les voyants de statut passent au vert
    \item Les valeurs affichées correspondent aux références utilisées
    \item La calibration est marquée comme "Calibré" dans l'interface
\end{itemize}

% =====================================
% RÉALISATION DES MESURES
% =====================================
\section{Réalisation des mesures}

\subsection{Préparation de la mesure}

\begin{enumerate}
    \item Vérifiez que la calibration est effectuée et valide
    \item Placez le plant de maïs dans le dispositif de mesure
    \item Assurez-vous que le plant est bien maintenu et stable
    \item Vérifiez l'affichage des valeurs en temps réel
\end{enumerate}

\subsection{Acquisition des données}

\begin{enumerate}
    \item Maintenez le bouton de la poignée du dispositif pendant toute la mesure.
    \item Le voyant orange en bas de la machine doit être allumé.
    \item Cliquez sur \textbf{Démarrer mesure} dans le logiciel
    \item Appliquez progressivement la force sur le plant
    \item Observez la courbe se dessiner en temps réel sur le graphique
    
    \item Continuez jusqu'à la rupture ou la limite souhaitée
    \item Cliquez sur \textbf{Arrêter mesure}
\end{enumerate}

\subsection{Paramètres de mesure}

\subsubsection{Fenêtre de moyennage}

Le système effectue une moyenne glissante sur les mesures pour réduire le bruit :
\begin{itemize}
    \item \textbf{Valeur par défaut :} 25 échantillons
    \item \textbf{Plage réglable :} 1 à 100 échantillons
    \item \textbf{Impact :} Plus la fenêtre est large, plus la courbe est lissée
\end{itemize}

Pour modifier ce paramètre :
\begin{enumerate}
    \item Accédez aux paramètres dans l'interface
    \item Ajustez la "Fenêtre de moyennage"
    \item Validez le changement
\end{enumerate}

\subsection{Qualité des mesures}

Pour obtenir des mesures de qualité :
\begin{itemize}
    \item Effectuez l'application de force de manière progressive et constante
    \item Évitez les à-coups et les vibrations
    \item Surveillez la stabilité des valeurs avant de commencer
    \item Arrêtez la mesure dès que la rupture est atteinte
\end{itemize}

% =====================================
% ANALYSE ET EXPORT DES DONNÉES
% =====================================
\section{Analyse et export des données}

\subsection{Visualisation des résultats}

Le graphique principal affiche :
\begin{itemize}
    \item \textbf{Axe X :} Angle d'inclinaison (en degrés)
    \item \textbf{Axe Y :} Force appliquée (en kg)
    \item \textbf{Courbe :} Évolution de la force en fonction de l'angle
\end{itemize}

Les informations statistiques incluent :
\begin{itemize}
    \item Durée totale de la mesure
    \item Nombre de points acquis
    \item Angle maximum atteint
    \item Force maximum appliquée
    \item Amplitude des mesures
\end{itemize}

\subsection{Export des données}

\subsubsection{Format Excel}

L'export Excel génère un fichier contenant :

\begin{itemize}
    \item \textbf{Feuille "Mesures MEVEM" :}
    \begin{itemize}
        \item Timestamp (temps relatif)
        \item Angle (degrés)
        \item Force (kg)
        \item Valeurs brutes des capteurs
        \item Nombre d'échantillons moyennés
    \end{itemize}
    
    \item \textbf{Feuille "Métadonnées" :}
    \begin{itemize}
        \item Date et heure de la mesure
        \item Paramètres de calibration
        \item Statistiques de la session
        \item Configuration du système
    \end{itemize}
\end{itemize}

\subsubsection{Procédure d'export}

\begin{enumerate}
    \item Après avoir terminé une mesure, cliquez sur \textbf{Exporter Excel}
    \item Choisissez l'emplacement de sauvegarde
    \item Le fichier est nommé automatiquement avec la date et l'heure
    \item Format : \texttt{mevem\_mesure\_AAAAMMJJ\_HHMMSS.xlsx}
\end{enumerate}

% =====================================
% RÉSOLUTION DE PROBLÈMES
% =====================================
\section{Résolution de problèmes}

\subsection{Problèmes de connexion}

\subsubsection{Le capteur n'est pas détecté}

\textbf{Symptômes :}
\begin{itemize}
    \item Voyant de connexion rouge
    \item Message "Aucun capteur détecté"
    \item Impossible de démarrer une mesure
\end{itemize}

\textbf{Solutions :}
\begin{enumerate}
    \item Vérifiez la connexion USB du dispositif
    \item Redémarrez l'application MEVEM
    \item Testez sur un autre port USB
    \item Vérifiez les permissions (Linux) :
    \begin{lstlisting}[language=bash]
ls -l /dev/ttyUSB*
groups $USER | grep dialout
    \end{lstlisting}
\end{enumerate}

\subsubsection{Port série occupé}

\textbf{Symptômes :}
\begin{itemize}
    \item Message "Port occupé par une autre application"
    \item Connexion intermittente
\end{itemize}

\textbf{Solutions :}
\begin{enumerate}
    \item Fermez toutes les applications utilisant des ports série
    \item Débranchez et rebranchez le dispositif USB
    \item Sur Linux, vérifiez les processus utilisant le port :
    \begin{lstlisting}[language=bash]
sudo lsof /dev/ttyUSB0
    \end{lstlisting}
\end{enumerate}

\subsection{Problèmes de calibration}

\subsubsection{Échec de la calibration}

\textbf{Symptômes :}
\begin{itemize}
    \item Message d'erreur pendant la procédure
    \item Valeurs aberrantes après calibration
    \item Impossibilité de terminer la calibration
\end{itemize}

\textbf{Solutions :}
\begin{enumerate}
    \item Vérifiez la stabilité du dispositif pendant la calibration
    \item Assurez-vous que les étalons sont corrects (0°, 45°, 1 kg)
    \item Répétez la procédure de calibration
    \item Vérifiez l'absence de vibrations durant la calibration
\end{enumerate}

\subsubsection{Dérive des mesures}

\textbf{Symptômes :}
\begin{itemize}
    \item Valeurs qui dérivent au fil du temps
    \item Différences importantes entre mesures successives
    \item Zéro instable
\end{itemize}

\textbf{Solutions :}
\begin{enumerate}
    \item Laissez le système se stabiliser 5-10 minutes après démarrage
    \item Refaites une calibration complète
    \item Vérifiez la température ambiante (évitez les variations)
    \item Contrôlez l'absence de courants d'air
\end{enumerate}

\subsection{Problèmes d'interface}

\subsubsection{L'interface web ne se charge pas}

\textbf{Solutions :}
\begin{enumerate}
    \item Accédez manuellement à \url{http://127.0.0.1:5000}
    \item Vérifiez que le port 5000 n'est pas utilisé par une autre application
    \item Redémarrez l'application MEVEM
    \item Testez avec un autre navigateur web
\end{enumerate}

\subsubsection{Graphique figé ou non mis à jour}

\textbf{Solutions :}
\begin{enumerate}
    \item Actualisez la page web (F5)
    \item Vérifiez la connexion WebSocket dans la console du navigateur
    \item Redémarrez l'application
    \item Cliquez sur \textbf{Actualiser statut}
\end{enumerate}

\subsection{Problèmes de performance}

\subsubsection{Acquisition lente ou saccadée}

\textbf{Causes possibles :}
\begin{itemize}
    \item Processeur surchargé
    \item Fenêtre de moyennage trop importante
    \item Problèmes de communication série
\end{itemize}

\textbf{Solutions :}
\begin{enumerate}
    \item Fermez les applications inutiles
    \item Réduisez la fenêtre de moyennage
    \item Utilisez un câble USB de meilleure qualité
    \item Évitez les hubs USB non alimentés
\end{enumerate}

% =====================================
% MAINTENANCE
% =====================================
\section{Maintenance}

\subsection{Maintenance préventive}

\subsubsection{Vérifications régulières}

\textbf{Avant chaque utilisation :}
\begin{itemize}
    \item Vérification visuelle du dispositif
    \item Test de la connexion USB
    \item Contrôle des valeurs au repos
\end{itemize}

\textbf{Après chaque utilisation :}
\begin{itemize}
    \item Nettoyage du dispositif avec un chiffon sec
\end{itemize}

\textbf{Mensuelle :}
\begin{itemize}
    \item Calibration complète du système
    \item Mise à jour du logiciel si disponible
    \item Archivage des données importantes
\end{itemize}

\subsection{Maintenance curative}

\subsubsection{Nettoyage du dispositif}

\begin{enumerate}
    \item Débranchez le dispositif de l'alimentation USB
    \item Nettoyez les surfaces avec un chiffon légèrement humide
    \item Évitez tout liquide près des connecteurs électroniques
    \item Séchez complètement avant reconnexion
\end{enumerate}

% =====================================
% CARACTÉRISTIQUES TECHNIQUES
% =====================================
\section{Caractéristiques techniques}

\subsection{Spécifications des capteurs}

\begin{table}[H]
\centering
\begin{tabular}{|l|l|l|}
\hline
\textbf{Paramètre} & \textbf{Capteur d'angle} & \textbf{Capteur de force} \\
\hline
Plage de mesure & 0° à 90° & 0 à 10 kg \\
\hline
Résolution & 0,1° & 1 g \\
\hline
Précision & ±0,5° & ±0,5\% PE \\
\hline
Linéarité & <1\% PE & <0,5\% PE \\
\hline
Dérive thermique & <0,1°/°C & <0,01\%/°C \\
\hline
Fréquence d'acquisition & \multicolumn{2}{|c|}{100 Hz} \\
\hline
\end{tabular}
\caption{Spécifications des capteurs}
\label{tab:specs_capteurs}
\end{table}

\subsection{Interface de communication}

\begin{table}[H]
\centering
\begin{tabular}{|l|l|}
\hline
\textbf{Paramètre} & \textbf{Valeur} \\
\hline
Interface & USB 2.0 (série virtuelle) \\
\hline
Vitesse & 115200 bauds \\
\hline
Format des données & 8 bits, pas de parité, 1 bit de stop \\
\hline
Protocole & VeTiMa, iMa, Ta \\
\hline
Longueur du câble & 2 mètres \\
\hline
\end{tabular}
\caption{Spécifications de communication}
\label{tab:specs_comm}
\end{table}

\subsection{Alimentation et consommation}

\begin{table}[H]
\centering
\begin{tabular}{|l|l|}
\hline
\textbf{Paramètre} & \textbf{Valeur} \\
\hline
Alimentation & 5V via USB \\
\hline
Consommation nominale & 200 mA \\
\hline
Consommation max & 350 mA \\
\hline
\end{tabular}
\caption{Caractéristiques électriques}
\label{tab:specs_elec}
\end{table}

% =====================================
% CONTACT ET SUPPORT
% =====================================
\section{Contact et support}

\subsection{Support technique}

Pour toute question technique ou problème d'utilisation :

\begin{itemize}
    \item \textbf{Email :} nolan.bayon@gmail.com
    \item \textbf{Téléphone :} +33 (0)7 83 90 70 45
\end{itemize}

% =====================================
% ANNEXES
% =====================================
\section{Annexes}

\subsection{Annexe A : Messages d'erreur}

\begin{longtable}{|p{4cm}|p{5cm}|p{5cm}|}
\hline
\textbf{Message} & \textbf{Cause} & \textbf{Solution} \\
\hline
\endhead
"Permission refusée" & Droits insuffisants sur le port série & Ajouter l'utilisateur au groupe dialout \\
\hline
"Port occupé" & Autre application utilise le port & Fermer les applications concurrentes \\
\hline
"Calibration échouée" & Instabilité pendant la calibration & Répéter avec plus de stabilité \\
\hline
"Connexion perdue" & Débranchement du dispositif & Vérifier la connexion USB \\
\hline
"Valeurs aberrantes" & Problème de capteur & Recalibrer ou contacter le support \\
\hline
\end{longtable}

\subsection{Annexe B : Formats de données}

\subsubsection{Format du fichier de calibration}

\begin{lstlisting}
{
  "angle": {
    "raw_min": 0,
    "raw_max": 1023,
    "real_min": 0.0,
    "real_max": 45.0,
    "calibrated": true
  },
  "force": {
    "raw_min": 0,
    "raw_max": 1023,
    "real_min": 0.0,
    "real_max": 1.0,
    "calibrated": true
  }
}
\end{lstlisting}

\subsection{Annexe C : Raccourcis clavier}

\begin{table}[H]
\centering
\begin{tabular}{|l|l|}
\hline
\textbf{Raccourci} & \textbf{Action} \\
\hline
F5 & Actualiser l'interface \\
\hline
Ctrl+S & Exporter les données \\
\hline
Espace & Démarrer/Arrêter la mesure \\
\hline
Échap & Annuler l'opération en cours \\
\hline
\end{tabular}
\caption{Raccourcis clavier disponibles}
\label{tab:shortcuts}
\end{table}

\end{document}
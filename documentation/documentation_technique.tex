\documentclass[12pt,a4paper]{article}
\usepackage[utf8]{inputenc}
\usepackage[french]{babel}
\usepackage[T1]{fontenc}
\usepackage{geometry}
\usepackage{graphicx}
\usepackage{fancyhdr}
\usepackage{amsmath}
\usepackage{amsfonts}
\usepackage{amssymb}
\usepackage{color}
\usepackage{hyperref}
\usepackage{listings}
\usepackage{float}
\usepackage{caption}
\usepackage{subcaption}
\usepackage{booktabs}
\usepackage{array}
\usepackage{multirow}
\usepackage{longtable}
\usepackage{xcolor}
\usepackage{tcolorbox}
% \usepackage{algorithmic}  % Package non disponible
% \usepackage{algorithm}   % Package non disponible

% Configuration des pages
\geometry{left=2.5cm,right=2.5cm,top=2.5cm,bottom=2.5cm}
\pagestyle{fancy}
\fancyhf{}
\rhead{\thepage}
\lhead{MEVEM - Documentation technique}

% Configuration des liens
\hypersetup{
    colorlinks=true,
    linkcolor=blue,
    filecolor=magenta,      
    urlcolor=cyan,
    pdftitle={MEVEM - Documentation technique},
    pdfauthor={MEVEM Team}
}

% Configuration des listings de code
\lstset{
    basicstyle=\ttfamily\small,
    breaklines=true,
    frame=single,
    numbers=left,
    numberstyle=\tiny,
    numbersep=5pt,
    backgroundcolor=\color{gray!10},
    keywordstyle=\color{blue},
    commentstyle=\color{green!60!black},
    stringstyle=\color{red},
    showstringspaces=false
}

% Configuration des boîtes colorées
\newtcolorbox{infobox}[1][]{
    colback=blue!5,
    colframe=blue!50!black,
    title=#1,
    fonttitle=\bfseries
}

\newtcolorbox{warningbox}[1][]{
    colback=orange!5,
    colframe=orange!50!black,
    title=#1,
    fonttitle=\bfseries
}

\newtcolorbox{codebox}[1][]{
    colback=gray!5,
    colframe=gray!50!black,
    title=#1,
    fonttitle=\bfseries\ttfamily
}

\title{%
    \vspace{-2cm}
    {\Huge \textbf{MEVEM}}\\[0.5cm]
    {\Large Mesure de la verse du maïs}\\[0.3cm]
    {\Large \textbf{DOCUMENTATION TECHNIQUE}}\\[0.5cm]
    {\normalsize Version 1.0}
}
\author{Nolan Bayon}
\date{\today}

\begin{document}

\maketitle
\thispagestyle{empty}

\newpage
\tableofcontents
\newpage

% =====================================
% ARCHITECTURE GÉNÉRALE
% =====================================
\section{Architecture générale}

\subsection{Vue d'ensemble du système}

Le système MEVEM repose sur une architecture client-serveur moderne combinant :
\begin{itemize}
    \item Un serveur web Flask intégré pour l'interface utilisateur
    \item Un module de communication série pour l'acquisition de données
    \item Une interface web responsive utilisant WebSocket pour les communications temps réel
    \item Un système de calibration et de traitement de données intégré
\end{itemize}

\begin{figure}[H]
    \centering
    \fbox{\parbox{0.9\textwidth}{\centering [DIAGRAMME: Architecture système] \\ 
    Schéma montrant les composants et leurs interactions}}
    \caption{Architecture générale du système MEVEM}
    \label{fig:architecture}
\end{figure}

\subsection{Composants logiciels principaux}

\subsubsection{Modules Python}

\begin{table}[H]
\centering
\begin{tabular}{|l|l|p{7cm}|}
\hline
\textbf{Module} & \textbf{Fichier} & \textbf{Fonction} \\
\hline
Application principale & \texttt{app.py} & Serveur web Flask, API REST, WebSocket \\
\hline
Communication série & \texttt{main.py} & Décodage capteurs, calibration, acquisition \\
\hline
Interface web & \texttt{templates/} & Interface utilisateur HTML/CSS/JS \\
\hline
Scripts de build & \texttt{build*.py} & Génération d'exécutables multi-plateforme \\
\hline
\end{tabular}
\caption{Modules logiciels principaux}
\label{tab:modules}
\end{table}

\subsubsection{Dépendances externes}

\begin{table}[H]
\centering
\begin{tabular}{|l|l|p{7cm}|}
\hline
\textbf{Bibliothèque} & \textbf{Version} & \textbf{Usage} \\
\hline
Flask & 2.3+ & Serveur web et API REST \\
\hline
Flask-SocketIO & 5.3+ & Communication WebSocket temps réel \\
\hline
PySerial & 3.5+ & Communication avec les capteurs série \\
\hline
Pandas & 2.0+ & Traitement et export des données \\
\hline
OpenPyXL & 3.1+ & Génération de fichiers Excel \\
\hline
PyInstaller & 5.13+ & Création d'exécutables autonomes \\
\hline
\end{tabular}
\caption{Dépendances logicielles}
\label{tab:dependencies}
\end{table}

% =====================================
% MODULE DE COMMUNICATION SÉRIE
% =====================================
\section{Module de communication série}

\subsection{Architecture du module}

Le module \texttt{main.py} implémente la classe \texttt{CalibratedSensorDecoder} qui gère :
\begin{itemize}
    \item La communication avec les capteurs via port série
    \item Le décodage des protocoles de données (VeTiMa, iMa, Ta)
    \item Le système de calibration interactif
    \item La conversion des valeurs brutes en unités physiques
\end{itemize}

\subsection{Protocoles de communication}

\subsubsection{Formats de données supportés}

Le système supporte trois protocoles de données :

\begin{codebox}[Protocoles série]
\begin{itemize}
    \item \textbf{VeTiMa} : \texttt{VeTiMa 0xXXXX 0xYYYY}
    \item \textbf{iMa} : \texttt{iMa 0xXXXX 0xYYYY}
    \item \textbf{Ta} : \texttt{Ta 0xXXXX 0xYYYY}
\end{itemize}
\end{codebox}

Où :
\begin{itemize}
    \item \texttt{0xXXXX} = valeur hexadécimale du capteur de force
    \item \texttt{0xYYYY} = valeur hexadécimale du capteur d'angle
\end{itemize}

\subsubsection{Analyse des trames}

\begin{lstlisting}[language=python, caption=Analyse des protocoles série]
# Patterns regex pour l'analyse des trames
self.patterns = {
    'VeTiMa': re.compile(r'VeTiMa\s*(0x[0-9A-Fa-f]{1,4})\s*(0x[0-9A-Fa-f]{1,4})'),
    'iMa': re.compile(r'iMa\s*(0x[0-9A-Fa-f]{1,4})\s*(0x[0-9A-Fa-f]{1,4})'),
    'Ta': re.compile(r'Ta\s*(0x[0-9A-Fa-f]{1,4})\s*(0x[0-9A-Fa-f]{1,4})')
}

def parse_line_raw(self, line):
    """Parse une ligne et retourne les valeurs brutes"""
    results = []
    for pattern_name, pattern in self.patterns.items():
        matches = pattern.findall(line)
        for match in matches:
            val1 = int(match[0], 16)  # Force
            val2 = int(match[1], 16)  # Angle
            
            if val1 <= 0xFFFF and val2 <= 0xFFFF:
                results.append({
                    'type': pattern_name,
                    'raw_force': val1,
                    'raw_angle': val2
                })
    return results if results else None
\end{lstlisting}

\subsection{Système de calibration}

\subsubsection{Structure des données de calibration}

\begin{lstlisting}[caption=Format de calibration]
{
  "angle": {
    "raw_min": 0,      // Valeur brute a 0 degres
    "raw_max": 1023,   // Valeur brute a 45 degres  
    "real_min": 0.0,   // Valeur reelle a 0 degres
    "real_max": 45.0,  // Valeur reelle a 45 degres
    "calibrated": true
  },
  "force": {
    "raw_min": 0,      // Valeur brute a vide
    "raw_max": 1023,   // Valeur brute a 1kg
    "real_min": 0.0,   // Valeur reelle a vide
    "real_max": 1.0,   // Valeur reelle a 1kg
    "calibrated": true
  }
}
\end{lstlisting}

\subsubsection{Algorithme de conversion}

\begin{tcolorbox}[title=Algorithme: Conversion des valeurs brutes en valeurs physiques]
\textbf{Entrées:} Valeur brute $v_{raw}$, paramètres de calibration $cal$\\
\textbf{Sortie:} Valeur physique $v_{phys}$

\begin{enumerate}
    \item \textbf{Si} $cal.calibrated = true$ \textbf{alors}
    \begin{enumerate}
        \item $ratio \leftarrow \frac{v_{raw} - cal.raw_{min}}{cal.raw_{max} - cal.raw_{min}}$
        \item $v_{phys} \leftarrow cal.real_{min} + ratio \times (cal.real_{max} - cal.real_{min})$
    \end{enumerate}
    \item \textbf{Sinon}
    \begin{enumerate}
        \item $v_{phys} \leftarrow v_{raw} \times \frac{scale_{default}}{resolution_{max}}$
    \end{enumerate}
    \item \textbf{Retourner} $v_{phys}$
\end{enumerate}
\end{tcolorbox}

\begin{lstlisting}[language=python, caption=Implémentation de la conversion]
def convert_raw_to_physical(self, raw_angle, raw_force):
    """Convertir les valeurs brutes en valeurs physiques"""
    # Conversion angle
    if self.calibration['angle']['calibrated']:
        angle_cal = self.calibration['angle']
        if angle_cal['raw_max'] != angle_cal['raw_min']:
            angle_ratio = (raw_angle - angle_cal['raw_min']) / \
                         (angle_cal['raw_max'] - angle_cal['raw_min'])
            angle_deg = angle_cal['real_min'] + \
                       angle_ratio * (angle_cal['real_max'] - angle_cal['real_min'])
        else:
            angle_deg = angle_cal['real_min']
    else:
        # Conversion par défaut
        angle_deg = raw_angle * 360.0 / 1023.0

    # Conversion force (similaire pour la force)
    # ... code similaire pour la force
    
    return angle_deg, force_kg
\end{lstlisting}

% =====================================
% SERVEUR WEB ET API
% =====================================
\section{Serveur web et API}

\subsection{Architecture Flask}

L'application utilise Flask comme framework web avec les extensions suivantes :
\begin{itemize}
    \item \textbf{Flask-SocketIO} : Communication bidirectionnelle temps réel
    \item \textbf{Threading} : Gestion des tâches d'acquisition en arrière-plan
    \item \textbf{CORS} : Support cross-origin pour l'interface web
\end{itemize}

\subsection{API REST}

\subsubsection{Endpoints de gestion des ports}

\begin{table}[H]
\centering
\begin{tabular}{|l|l|p{7cm}|}
\hline
\textbf{Endpoint} & \textbf{Méthode} & \textbf{Description} \\
\hline
\texttt{/api/ports/list} & GET & Liste les ports série disponibles \\
\hline
\texttt{/api/ports/select} & POST & Sélectionne un port série spécifique \\
\hline
\end{tabular}
\caption{API de gestion des ports}
\end{table}

\subsubsection{Endpoints de calibration}

\begin{table}[H]
\centering
\begin{tabular}{|l|l|p{7cm}|}
\hline
\textbf{Endpoint} & \textbf{Méthode} & \textbf{Description} \\
\hline
\texttt{/api/calibration/status} & GET & État actuel de la calibration \\
\hline
\texttt{/api/calibration/start} & POST & Démarre la procédure de calibration \\
\hline
\end{tabular}
\caption{API de calibration}
\end{table}

\subsubsection{Endpoints de mesure}

\begin{table}[H]
\centering
\begin{tabular}{|l|l|p{7cm}|}
\hline
\textbf{Endpoint} & \textbf{Méthode} & \textbf{Description} \\
\hline
\texttt{/api/measurement/start} & POST & Démarre l'acquisition de données \\
\hline
\texttt{/api/measurement/stop} & POST & Arrête l'acquisition de données \\
\hline
\texttt{/api/measurement/data} & GET & Récupère les données de mesure \\
\hline
\texttt{/api/measurement/clear} & POST & Efface les données actuelles \\
\hline
\texttt{/api/measurement/export/excel} & POST & Exporte les données en Excel \\
\hline
\end{tabular}
\caption{API de mesure}
\end{table}

\subsubsection{Endpoints de configuration}

\begin{table}[H]
\centering
\begin{tabular}{|l|l|p{7cm}|}
\hline
\textbf{Endpoint} & \textbf{Méthode} & \textbf{Description} \\
\hline
\texttt{/api/averaging/get} & GET & Récupère la fenêtre de moyennage \\
\hline
\texttt{/api/averaging/set} & POST & Définit la fenêtre de moyennage \\
\hline
\end{tabular}
\caption{API de configuration}
\end{table}

\subsection{Communication WebSocket}

\subsubsection{Événements WebSocket}

\begin{table}[H]
\centering
\begin{tabular}{|l|l|p{7cm}|}
\hline
\textbf{Événement} & \textbf{Direction} & \textbf{Description} \\
\hline
\texttt{connect} & Client → Serveur & Connexion du client \\
\hline
\texttt{disconnect} & Client → Serveur & Déconnexion du client \\
\hline
\texttt{connected} & Serveur → Client & Confirmation de connexion \\
\hline
\texttt{measurement\_data} & Serveur → Client & Données de mesure temps réel \\
\hline
\texttt{error} & Serveur → Client & Messages d'erreur \\
\hline
\end{tabular}
\caption{Événements WebSocket}
\end{table}

\subsubsection{Structure des données temps réel}

\begin{lstlisting}[language=json, caption=Format des données WebSocket]
{
  "timestamp": 1.234,      // Temps relatif en secondes
  "angle": 12.5,           // Angle en degrés
  "force": 0.245,          // Force en kg
  "raw_angle": 512,        // Valeur brute angle
  "raw_force": 256,        // Valeur brute force
  "samples_count": 25      // Nombre d'échantillons moyennés
}
\end{lstlisting}

% =====================================
% TRAITEMENT DES DONNÉES
% =====================================
\section{Traitement des données}

\subsection{Système de moyennage}

Le système implémente un moyennage glissant pour réduire le bruit des mesures :

\begin{infobox}[Principe du moyennage]
Les valeurs brutes des capteurs sont accumulées dans des tampons circulaires. Quand le nombre d'échantillons atteint la fenêtre configurée, une moyenne est calculée et transmise.
\end{infobox}

\begin{lstlisting}[language=python, caption=Implémentation du moyennage]
# Configuration du moyennage
averaging_window = 25  # Nombre de valeurs pour la moyenne
angle_accumulator = []  # Accumulateur pour les angles
force_accumulator = []  # Accumulateur pour les forces

def measurement_worker():
    """Worker thread pour la mesure en continu"""
    global current_measurement, measurement_active, decoder
    global averaging_window, angle_accumulator, force_accumulator
    
    while measurement_active:
        # Lecture des données série
        parsed = decoder.parse_line(line)
        
        if parsed:
            for data in parsed:
                # Accumuler les valeurs
                angle_accumulator.append({
                    'angle': data['angle_deg'],
                    'raw_angle': data['raw_angle']
                })
                force_accumulator.append({
                    'force': data['force_kg'],
                    'raw_force': data['raw_force']
                })
                
                # Si assez de valeurs, calculer la moyenne
                if len(angle_accumulator) >= averaging_window:
                    avg_angle = sum([item['angle'] for item in angle_accumulator]) / len(angle_accumulator)
                    avg_force = sum([item['force'] for item in force_accumulator]) / len(force_accumulator)
                    
                    # Créer le point de mesure
                    measurement_point = {
                        'timestamp': time.time() - start_time,
                        'angle': round(avg_angle, 2),
                        'force': round(avg_force, 3),
                        'raw_angle': int(avg_raw_angle),
                        'raw_force': int(avg_raw_force),
                        'samples_count': len(angle_accumulator)
                    }
                    
                    # Vider les accumulateurs
                    angle_accumulator = []
                    force_accumulator = []
\end{lstlisting}

\subsection{Export des données}

\subsubsection{Génération Excel}

Le système génère des fichiers Excel avec deux feuilles :

\begin{lstlisting}[language=python, caption=Génération du fichier Excel]
def export_to_excel():
    """Exporter les données vers Excel"""
    # Créer un DataFrame pandas
    df = pd.DataFrame(current_measurement)
    
    # Créer un buffer en mémoire
    output = io.BytesIO()
    
    # Écrire le fichier Excel
    with pd.ExcelWriter(output, engine='openpyxl') as writer:
        # Feuille des données
        df.to_excel(writer, sheet_name='Mesures MEVEM', index=False)
        
        # Feuille des métadonnées
        metadata_df = pd.DataFrame({
            'Information': [
                'Date de mesure', 
                'Nombre de points', 
                'Durée (s)', 
                'Angle min (°)', 
                'Angle max (°)', 
                'Force min (kg)', 
                'Force max (kg)'
            ],
            'Valeur': [
                datetime.now().strftime('%Y-%m-%d %H:%M:%S'),
                len(current_measurement),
                round(max([p['timestamp'] for p in current_measurement]), 2),
                # ... autres statistiques
            ]
        })
        metadata_df.to_excel(writer, sheet_name='Métadonnées', index=False)
\end{lstlisting}

% =====================================
% INTERFACE UTILISATEUR
% =====================================
\section{Interface utilisateur}

\subsection{Technologies web utilisées}

L'interface utilisateur utilise :
\begin{itemize}
    \item \textbf{HTML5} : Structure de la page
    \item \textbf{CSS3} : Mise en forme et responsive design
    \item \textbf{JavaScript ES6} : Logique client et interactions
    \item \textbf{Chart.js} : Graphiques interactifs temps réel
    \item \textbf{Socket.IO client} : Communication WebSocket
    \item \textbf{Bootstrap} : Framework CSS responsive
\end{itemize}

\subsection{Composants principaux}

\subsubsection{Graphique temps réel}

\begin{lstlisting}[caption=Configuration du graphique Chart.js]
// Configuration du graphique principal
const chartConfig = {
    type: 'scatter',
    data: {
        datasets: [{
            label: 'Force vs Angle',
            data: [],
            backgroundColor: 'rgba(54, 162, 235, 0.6)',
            borderColor: 'rgba(54, 162, 235, 1)',
            borderWidth: 2,
            pointRadius: 2,
            showLine: true,
            tension: 0.1
        }]
    },
    options: {
        responsive: true,
        maintainAspectRatio: false,
        scales: {
            x: {
                type: 'linear',
                position: 'bottom',
                title: {
                    display: true,
                    text: 'Angle (degrés)'
                }
            },
            y: {
                title: {
                    display: true,
                    text: 'Force (kg)'
                }
            }
        },
        animation: {
            duration: 0 // Pas d'animation pour le temps réel
        }
    }
};
\end{lstlisting}

\subsubsection{Gestion des événements WebSocket}

\begin{lstlisting}[caption=Gestion des WebSockets côté client]
// Connexion WebSocket
const socket = io();

// Réception des données de mesure
socket.on('measurement_data', function(data) {
    // Ajouter le point au graphique
    chart.data.datasets[0].data.push({
        x: data.angle,
        y: data.force
    });
    
    // Limiter le nombre de points affichés
    if (chart.data.datasets[0].data.length > maxPoints) {
        chart.data.datasets[0].data.shift();
    }
    
    // Mettre à jour le graphique
    chart.update('none');
    
    // Mettre à jour les statistiques
    updateStatistics(data);
});

// Gestion des erreurs
socket.on('error', function(error) {
    console.error('Erreur WebSocket:', error.message);
    showNotification('Erreur: ' + error.message, 'error');
});
\end{lstlisting}

% =====================================
% DÉPLOIEMENT ET BUILD
% =====================================
\section{Déploiement et build}

\subsection{Système de build multi-plateforme}

Le système utilise PyInstaller pour créer des exécutables autonomes :

\begin{codebox}[Scripts de build disponibles]
\begin{itemize}
    \item \texttt{build.py} : Build universel avec sélection de plateforme
    \item \texttt{build\_windows.py} : Build spécifique Windows
    \item \texttt{build\_linux.py} : Build spécifique Linux
    \item \texttt{build\_final.py} : Build optimisé pour production
\end{itemize}
\end{codebox}

\subsection{Configuration PyInstaller}

\begin{lstlisting}[language=python, caption=Configuration PyInstaller type]
def build_executable(platform='current'):
    """Construire l'exécutable pour la plateforme spécifiée"""
    
    # Configuration commune
    base_options = [
        '--name=mevem',
        '--onefile',
        '--windowed' if platform == 'windows' else '',
        '--add-data=templates;templates',
        '--add-data=static;static',
        '--hidden-import=eventlet',
        '--hidden-import=socketio',
        '--hidden-import=engineio',
        '--clean',
        'app.py'
    ]
    
    # Options spécifiques à la plateforme
    if platform == 'windows':
        base_options.extend([
            '--icon=icon.ico',
            '--version-file=version.txt',
            '--distpath=dist_windows'
        ])
    elif platform == 'linux':
        base_options.extend([
            '--distpath=dist_linux'
        ])
    
    # Exécuter PyInstaller
    PyInstaller.__main__.run(base_options)
\end{lstlisting}

\subsection{Gestion des dépendances}

\subsubsection{Fichier requirements.txt}

\begin{lstlisting}[caption=Dépendances Python]
Flask>=2.3.0
Flask-SocketIO>=5.3.0
pyserial>=3.5
pandas>=2.0.0
openpyxl>=3.1.0
pyinstaller>=5.13.0
eventlet>=0.33.0
python-socketio>=5.8.0
python-engineio>=4.7.0
\end{lstlisting}

\subsubsection{Optimisations de build}

\begin{warningbox}[Considérations de performance]
\begin{itemize}
    \item Les builds incluent un environnement Python complet (environ 50-80 MB)
    \item L'option \texttt{--onefile} ralentit le démarrage mais simplifie la distribution
    \item Les modules \texttt{eventlet} et \texttt{socketio} nécessitent des imports explicites
\end{itemize}
\end{warningbox}

% =====================================
% SÉCURITÉ ET PERFORMANCE
% =====================================
\section{Sécurité et performance}

\subsection{Mesures de sécurité}

\subsubsection{Sécurisation des communications}

\begin{itemize}
    \item \textbf{Interface locale uniquement} : Le serveur n'écoute que sur \texttt{127.0.0.1}
    \item \textbf{Pas d'exposition réseau} : Aucun accès distant par défaut
    \item \textbf{Validation des données} : Toutes les entrées utilisateur sont validées
    \item \textbf{Gestion des exceptions} : Gestion robuste des erreurs série et réseau
\end{itemize}

\subsubsection{Sécurisation des ports série}

\begin{lstlisting}[language=python, caption=Vérification de l'accès aux ports]
def check_port_access(port):
    """Vérifier l'accès à un port série"""
    try:
        # Essayer d'ouvrir le port brièvement
        test_conn = serial.Serial(port, timeout=0.1)
        test_conn.close()
        return {'accessible': True}
    except serial.SerialException as e:
        error_msg = str(e)
        if 'Permission denied' in error_msg:
            return {
                'accessible': False,
                'error': 'Permission refusée - Ajoutez votre utilisateur au groupe dialout'
            }
        elif 'Device or resource busy' in error_msg:
            return {
                'accessible': False,
                'error': 'Port occupé par une autre application'
            }
        # ... autres cas d'erreur
\end{lstlisting}

\subsection{Optimisations de performance}

\subsubsection{Gestion mémoire}

\begin{itemize}
    \item \textbf{Buffers circulaires} : Limitation de la taille des données en mémoire
    \item \textbf{Nettoyage automatique} : Libération des ressources après usage
    \item \textbf{Threads dédiés} : Séparation acquisition/interface pour éviter les blocages
\end{itemize}

\begin{lstlisting}[language=python, caption=Gestion des buffers circulaires]
from collections import deque

# Stockage des données avec taille limitée
self.data_buffer = deque(maxlen=1000)

# Limitation du nombre de points dans l'interface
if len(chart.data.datasets[0].data) > maxPoints:
    chart.data.datasets[0].data.shift();
\end{lstlisting}

\subsubsection{Optimisations réseau}

\begin{itemize}
    \item \textbf{Compression des données WebSocket} : Réduction de la bande passante
    \item \textbf{Mise à jour différentielle} : Envoi des changements uniquement
    \item \textbf{Debouncing} : Limitation de la fréquence des mises à jour interface
\end{itemize}

% =====================================
% TESTS ET VALIDATION
% =====================================
\section{Tests et validation}

\subsection{Stratégie de test}

\subsubsection{Types de tests implémentés}

\begin{table}[H]
\centering
\begin{tabular}{|l|l|p{7cm}|}
\hline
\textbf{Type de test} & \textbf{Couverture} & \textbf{Outils} \\
\hline
Tests unitaires & Fonctions de conversion, parsing & pytest, unittest \\
\hline
Tests d'intégration & Communication série, API REST & pytest-flask \\
\hline
Tests de performance & Charge, mémoire, latence & pytest-benchmark \\
\hline
Tests manuels & Interface utilisateur, workflows & Manuel \\
\hline
\end{tabular}
\caption{Stratégie de test}
\label{tab:test_strategy}
\end{table}

\subsubsection{Cas de tests critiques}

\begin{enumerate}
    \item \textbf{Communication série} : Test de tous les protocoles supportés
    \item \textbf{Calibration} : Vérification de la précision des conversions
    \item \textbf{Moyennage} : Validation de l'algorithme de moyennage glissant
    \item \textbf{Export} : Intégrité des fichiers Excel générés
    \item \textbf{WebSocket} : Stabilité des communications temps réel
\end{enumerate}

\subsection{Tests de performance}

\subsubsection{Benchmarks}

\begin{table}[H]
\centering
\begin{tabular}{|l|l|l|}
\hline
\textbf{Métrique} & \textbf{Valeur cible} & \textbf{Valeur mesurée} \\
\hline
Fréquence d'acquisition & 100 Hz & 95-105 Hz \\
\hline
Latence WebSocket & < 50 ms & 15-25 ms \\
\hline
Consommation mémoire & < 100 MB & 65-85 MB \\
\hline
Temps de démarrage & < 5 s & 2-3 s \\
\hline
\end{tabular}
\caption{Benchmarks de performance}
\label{tab:benchmarks}
\end{table}

\subsubsection{Tests de charge}

\begin{lstlisting}[language=python, caption=Test de charge WebSocket]
import asyncio
import websockets
import time

async def stress_test_websocket():
    """Test de charge sur les WebSockets"""
    start_time = time.time()
    
    # Simuler 1000 messages par seconde pendant 1 minute
    for i in range(60000):
        data = {
            'timestamp': time.time() - start_time,
            'angle': random.uniform(0, 45),
            'force': random.uniform(0, 1)
        }
        
        # Envoyer via WebSocket
        await websocket.send(json.dumps(data))
        
        if i % 1000 == 0:
            print(f"Envoyé {i} messages")
        
        await asyncio.sleep(0.001)  # 1000 Hz
\end{lstlisting}

% =====================================
% MAINTENANCE ET ÉVOLUTION
% =====================================
\section{Maintenance et évolution}

\subsection{Structure de versioning}

Le projet utilise le versioning sémantique (SemVer) :
\begin{itemize}
    \item \textbf{MAJOR} : Changements incompatibles de l'API
    \item \textbf{MINOR} : Fonctionnalités ajoutées de manière rétrocompatible
    \item \textbf{PATCH} : Corrections de bugs rétrocompatibles
\end{itemize}

\subsection{Logs et debugging}

\subsubsection{Système de logging}

\begin{lstlisting}[language=python, caption=Configuration du logging]
import logging
from logging.handlers import RotatingFileHandler

# Configuration du logger principal
logging.basicConfig(
    level=logging.INFO,
    format='%(asctime)s - %(name)s - %(levelname)s - %(message)s',
    handlers=[
        RotatingFileHandler('mevem.log', maxBytes=1024*1024, backupCount=5),
        logging.StreamHandler()
    ]
)

logger = logging.getLogger('MEVEM')

# Utilisation dans le code
logger.info("🌽 MEVEM - Démarrage de l'application")
logger.warning("⚠️ Mode démo activé - aucun capteur détecté")
logger.error("[ERREUR] Erreur de communication série")
\end{lstlisting}

\subsubsection{Niveaux de debugging}

\begin{table}[H]
\centering
\begin{tabular}{|l|l|p{7cm}|}
\hline
\textbf{Niveau} & \textbf{Usage} & \textbf{Contenu} \\
\hline
DEBUG & Développement & Détails de communication série, trames \\
\hline
INFO & Production & Événements importants, connexions \\
\hline
WARNING & Production & Problèmes non critiques, mode dégradé \\
\hline
ERROR & Production & Erreurs, exceptions gérées \\
\hline
CRITICAL & Production & Erreurs fatales, arrêt d'application \\
\hline
\end{tabular}
\caption{Niveaux de logging}
\label{tab:log_levels}
\end{table}

\subsection{Roadmap d'évolution}

\subsubsection{Version 1.1 (Q2 2024)}

\begin{itemize}
    \item Support de protocoles série additionnels
    \item Interface de configuration avancée
    \item Export vers formats CSV et JSON
    \item Mode batch pour traitement de fichiers
\end{itemize}

\subsubsection{Version 1.2 (Q3 2024)}

\begin{itemize}
    \item Interface mobile responsive
    \item API RESTful complète pour intégration
    \item Base de données pour historique des mesures
    \item Analyses statistiques avancées
\end{itemize}

\subsubsection{Version 2.0 (Q4 2024)}

\begin{itemize}
    \item Architecture microservices
    \item Support multi-utilisateur
    \item Interface cloud (optionnelle)
    \item Intégration IA pour analyse prédictive
\end{itemize}

% =====================================
% ANNEXES TECHNIQUES
% =====================================
\section{Annexes techniques}

\subsection{Annexe A : Schémas de base de données}

\subsubsection{Format des fichiers de calibration}

\begin{lstlisting}[caption=Structure future de la base de données]
-- Table des sessions de mesure
CREATE TABLE measurement_sessions (
    id INTEGER PRIMARY KEY AUTOINCREMENT,
    timestamp DATETIME DEFAULT CURRENT_TIMESTAMP,
    duration_seconds REAL,
    sample_count INTEGER,
    max_angle REAL,
    max_force REAL,
    calibration_used TEXT, -- JSON des paramètres de calibration
    notes TEXT
);

-- Table des points de mesure
CREATE TABLE measurement_points (
    id INTEGER PRIMARY KEY AUTOINCREMENT,
    session_id INTEGER REFERENCES measurement_sessions(id),
    timestamp REAL, -- Temps relatif dans la session
    angle_deg REAL,
    force_kg REAL,
    raw_angle INTEGER,
    raw_force INTEGER,
    samples_count INTEGER
);
\end{lstlisting}

\subsection{Annexe B : Protocoles de communication détaillés}

\subsubsection{Spécifications électriques}

\begin{table}[H]
\centering
\begin{tabular}{|l|l|l|}
\hline
\textbf{Paramètre} & \textbf{Valeur} & \textbf{Tolérance} \\
\hline
Tension d'alimentation & 5,0 V & ±5\% \\
\hline
Courant de repos & 50 mA & ±10 mA \\
\hline
Courant en mesure & 200 mA & ±50 mA \\
\hline
Résolution ADC & 12 bits & - \\
\hline
Fréquence d'échantillonnage & 100 Hz & ±1 Hz \\
\hline
\end{tabular}
\caption{Spécifications électriques détaillées}
\label{tab:electrical_specs}
\end{table}

\subsubsection{Timing des communications}

\begin{figure}[H]
\centering
\fbox{\parbox{0.9\textwidth}{\centering [DIAGRAMME: Timing] \\ 
 Chronogramme des communications série}}
\caption{Diagramme de timing des communications série}
\label{fig:timing}
\end{figure}

\subsection{Annexe C : Code source des utilitaires}

\subsubsection{Script de vérification système}

\begin{lstlisting}[language=python, caption=Utilitaire de diagnostic système]
#!/usr/bin/env python3
"""
Utilitaire de diagnostic système MEVEM
Vérifie la configuration et les prérequis
"""

import os
import sys
import platform
import serial.tools.list_ports
import grp
import pwd

def check_system_requirements():
    """Vérifier les prérequis système"""
    print("🔍 Vérification des prérequis système")
    print("=" * 50)
    
    # Version Python
    python_version = sys.version_info
    print(f"🐍 Python: {python_version.major}.{python_version.minor}.{python_version.micro}")
    
    if python_version < (3, 8):
        print("[ERREUR] Python 3.8+ requis")
        return False
    else:
        print("✅ Version Python compatible")
    
    # Système d'exploitation
    system = platform.system()
    print(f"💻 OS: {system} {platform.release()}")
    
    # Ports série disponibles
    ports = list(serial.tools.list_ports.comports())
    print(f"🔌 Ports série: {len(ports)} trouvés")
    
    for port in ports:
        print(f"   • {port.device} - {port.description}")
    
    # Permissions (Linux uniquement)
    if system == "Linux":
        check_linux_permissions()
    
    return True

def check_linux_permissions():
    """Vérifier les permissions Linux"""
    print("\n🔐 Vérification des permissions Linux")
    
    try:
        # Obtenir l'utilisateur actuel
        current_user = pwd.getpwuid(os.getuid()).pw_name
        print(f"👤 Utilisateur: {current_user}")
        
        # Vérifier l'appartenance au groupe dialout
        try:
            dialout_group = grp.getgrnam('dialout')
            if current_user in dialout_group.gr_mem:
                print("[OK] Utilisateur dans le groupe dialout")
            else:
                print("[ERREUR] Utilisateur PAS dans le groupe dialout")
                print(f"   Exécutez: sudo usermod -a -G dialout {current_user}")
        except KeyError:
            print("[AVERTISSEMENT] Groupe dialout non trouvé")
    
    except Exception as e:
        print(f"[ERREUR] Erreur vérification permissions: {e}")

if __name__ == "__main__":
    check_system_requirements()
\end{lstlisting}

\subsection{Annexe D : Configuration de développement}

\subsubsection{Configuration IDE recommandée}

\begin{codebox}[VS Code settings.json]
\begin{lstlisting}
{
    "python.defaultInterpreterPath": "./venv/bin/python",
    "python.formatting.provider": "black",
    "python.linting.enabled": true,
    "python.linting.pylintEnabled": true,
    "files.exclude": {
        "**/__pycache__": true,
        "**/*.pyc": true,
        "**/dist": true,
        "**/build": true
    },
    "editor.formatOnSave": true,
    "python.testing.pytestEnabled": true
}
\end{lstlisting}
\end{codebox}

\subsubsection{Configuration pre-commit}

\begin{lstlisting}[caption=.pre-commit-config.yaml]
repos:
  - repo: https://github.com/psf/black
    rev: 23.3.0
    hooks:
      - id: black
        language_version: python3.8
  
  - repo: https://github.com/pycqa/flake8
    rev: 6.0.0
    hooks:
      - id: flake8
        args: [--max-line-length=88]
  
  - repo: https://github.com/pycqa/isort
    rev: 5.12.0
    hooks:
      - id: isort
        args: [--profile, black]
\end{lstlisting}

\end{document}